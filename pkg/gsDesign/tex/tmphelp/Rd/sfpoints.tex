\HeaderA{sfPoints}{4.5: Pointwise Spending Function}{sfPoints}
\keyword{design}{sfPoints}
\begin{Description}\relax
The function \code{sfPoints} implements a spending function with values specified for an arbitrary set of specified points.
It is now recommended to use sfLinear rather than sfPoints.
Normally \code{sfPoints} will be passed to \code{gsDesign} in the parameter \code{sfu} for the upper bound or
\code{sfl} for the lower bound to specify a spending function family for a design.
In this case, the user does not need to know the calling sequence, just the points they wish to specify.
If using \code{sfPoints()} in a design, it is recommended to specify how to interpolate between the specified points (e.g,, linear interpolation); also consider fitting smooth spending functions; see 
\LinkA{Spending function overview}{Spending function overview}.
\end{Description}
\begin{Usage}
\begin{verbatim}
sfPoints(alpha, t, param)
\end{verbatim}
\end{Usage}
\begin{Arguments}
\begin{ldescription}
\item[\code{alpha}] Real value \eqn{> 0}{} and no more than 1. Normally, \code{alpha=0.025} for one-sided Type I error specification
or \code{alpha=0.1} for Type II error specification. However, this could be set to 1 if for descriptive purposes
you wish to see the proportion of spending as a function of the proportion of sample size/information.
\item[\code{t}] A vector of points with increasing values from 0 to 1, inclusive. The last point should be 1.
Values of the proportion of sample size/information for which the spending function will be computed.
\item[\code{param}] A vector of the same length as \code{t} specifying the cumulative proportion of spending
to corresponding to each point in \code{t}.
\end{ldescription}
\end{Arguments}
\begin{Value}
An object of type \code{spendfn}. See spending functions for further details.
\end{Value}
\begin{Note}\relax
The manual is not linked to this help file, but is available in library/gsdesign/doc/gsDesignManual.pdf
in the directory where R is installed.
\end{Note}
\begin{Author}\relax
Keaven Anderson \email{keaven\_anderson@merck.}
\end{Author}
\begin{References}\relax
Jennison C and Turnbull BW (2000), \emph{Group Sequential Methods with Applications to Clinical Trials}.
Boca Raton: Chapman and Hall.
\end{References}
\begin{SeeAlso}\relax
\LinkA{Spending function overview}{Spending function overview}, \code{\LinkA{gsDesign}{gsDesign}}, \LinkA{gsDesign package overview}{gsDesign package overview}, \LinkA{sfLogistic}{sfLogistic}
\end{SeeAlso}
\begin{Examples}
\begin{ExampleCode}
# example to specify spending on a pointwise basis
x <- gsDesign(k=6, sfu=sfPoints, sfupar=c(.01, .05, .1, .25, .5, 1),
              test.type=2)
x

# get proportion of upper spending under null hypothesis
# at each analysis
y <- x$upper$prob[, 1] / .025

# change to cumulative proportion of spending
for(i in 2:length(y)) 
    y[i] <- y[i - 1] + y[i]

# this should correspond to input sfupar
round(y, 6)

# plot these cumulative spending points
plot(1:6/6, y, main="Pointwise spending function example", 
    xlab="Proportion of final sample size", 
    ylab="Cumulative proportion of spending", 
    type="p")

# approximate this with a t-distribution spending function
# by fitting 3 points
tx <- 0:100/100
lines(tx, sfTDist(1, tx, c(c(1, 3, 5)/6, .01, .1, .5))$spend)
text(x=.6, y=.9, labels="Pointwise Spending Approximated by")
text(x=.6, y=.83, "t-Distribution Spending with 3-point interpolation")
\end{ExampleCode}
\end{Examples}

