\subsection{Standardized treatment effect}
The common use of \texttt{gsDesign()} is to specify this fixed design size rather than a treatment effect to generate a design.
However, it is possible to generate a design by inputting a standardized treatment effect.
Let $\Phi^{-1}()$ denote the inverse of
the cumulative standard normal distribution function. In order for the fixed
design with level $\alpha$ (1-sided), to have power 100(1--$\beta$)\% to
reject $\theta$=0 when in fact $\theta=\delta$, we must have
\begin{equation}
I_{fix}=\left(  \frac{\Phi^{-1}(1-\alpha)+\Phi^{-1}(1-\beta)}{\delta}\right)
^{2}.\label{fixed design sample size}%
\end{equation}
From (\ref{ssratio}) and (\ref{fixed design sample size}), if we let
$\delta=\Phi^{-1}(1-\alpha)+\Phi^{-1}(1-\beta)$ then $r=I_{K}$. Because of
this relation, this is the value of $\delta$ that is used and displayed when
information ratios are computed in \texttt{gsDesign()}.
