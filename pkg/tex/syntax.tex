\section{Syntax}

The two primary functions available in this package are:

\begin{itemize}
\item \texttt{gsDesign(argument1 = value1, \ldots)}

\item \texttt{gsProbability(argument1 = value1, \ldots)}
\end{itemize}

All arguments are in lower case with the exception of \texttt{n.I} in
\texttt{gsProbability()}. Although there are many arguments available in
\texttt{gsDesign()}, many of these need not be specified as the defaults are adequate. Returned values from \texttt{gsDesign()} and
\texttt{gsProbability()} are objects from newly defined classes named 
\texttt{gsDesign} and \texttt{gsProbability}. The \texttt{gsDesign} class is 
an extension of (inherits the characteristics of) the \texttt{gsProbability} 
class. These classes are described in
Section~\ref{sec:objtypes}, The \texttt{gsDesign} and \texttt{gsProbability} 
Object Types. The output functions \texttt{print()}, and \texttt{plot()} 
for the \texttt{gsDesign} and \texttt{gsProbability} classes are described 
further in Section~\ref{sec:formatted}, Formatted Output.

In the following, there is a single parameter $\theta$ for which we generally
are trying to test a null hypothesis H$_{0}$: $\theta=0$ against some
alternative such as H1: $\theta$ $\neq0$. In general, the parameter $\theta$
is a standardized treatment difference and the statistic $Z_{n}$
for testing after $n$ observations have a distribution that is well
approximated by a normal distribution with mean $\sqrt{n}\theta$ and variance
1. See Jennison and Turnbull \cite{JTBook} for extensive discussions of
various types of endpoints for which the approximation is reasonable. For
example, normal, binomial and time-to-event endpoints may be considered.
Section~\ref{sec:statmethods}, Statistical Methods, provides more detail.

As a supplement to the primary functions \texttt{gsDesign()} and
\texttt{gsProbability()}, we also consider functions \texttt{gsCP()} and
\texttt{gsBoundCP()}\ which compute the conditional probabilities of boundary
crossing. Background for conditional power computations is also provided in 
Section~\ref{sec:statmethods}, Statistical Methods.

Finally, some utility functions for working with two-arm binomial or survival
trials are provided. For the binomial distribution there are functions for
fixed (non-group-sequential) designs that compute sample size, perform
statistical testing, and perform simulations. For survival, a sample size
function is provided.

\subsection{gsDesign() syntax}

As noted above, \texttt{gsDesign()} provides sample size and boundaries for a
group sequential design based on treatment effect, spending functions for
boundary crossing probabilities, and relative timing of each analysis. The
most general form of the call to \texttt{gsDesign()} with default arguments is:

\bigskip

\texttt{gsDesign(k=3, test.type=4, alpha=0.025, beta=0.1, astar=0, delta=0,
n.fix=1,}

\texttt{\ \ \ timing=1, sfu=sfHSD, sfupar= -4, sfl=sfHSD, sflpar= -2,
tol=0.000001, r=18}

\texttt{\ \ \ n.I=0,maxn.IPlan=0)}.

\bigskip

The arguments are as follows:

\begin{itemize}
\item \texttt{k} = integer (\texttt{k}%
%TCIMACRO{\TEXTsymbol{>}}%
%BeginExpansion
$>$%
%EndExpansion
1). Specifies the number of analyses, including interim and final. The default
value is 3. {\bf Note:} Use of very large \texttt{k} produces an error
message. How large varies with the value of \texttt{test.type} specified. For
example, with default arguments (\texttt{test.type=3}), specifying \texttt{k}
%TCIMACRO{\TEXTsymbol{>} }%
%BeginExpansion
$>$
%EndExpansion
23 produces the error message "False negative rates not achieved." For
other values of \texttt{test.type}, the upper limit is larger.

\item \texttt{test.type} = integer (1$\leq$ \texttt{test.type} $\leq$ 6). In
the following we denote by $\theta$ the parameter we are testing for. The
null hypothesis will always be H$_{0}$: $\theta=0$. The alternative hypothesis
of interest varies with the value of \texttt{test.type}.

\begin{description}
\item[=1:] One-sided testing. Tests the null hypothesis H$_{0}$: $\theta=0$
against the alternative hypothesis H$_{1}$: $\theta>0$. The user specifies only
upper boundary crossing probabilities under the null hypothesis. There is no
lower boundary.

\item[=2:] Symmetric, two-sided testing. Tests H$_{0}$: $\theta = 0$ against the
alternative hypothesis H1: $\theta \neq 0$. Boundary crossing probabilities
are specified under the null hypothesis. The user specifies only upper boundary
crossing probabilities

\item[=3:] Asymmetric two-sided testing with binding lower bound and beta
spending. Asymmetric two-sided testing is also often referred to as two
one-sided tests in that testing is done both for efficacy (attempts to reject
H$_{0}$: $\theta = 0$ in favor of H$_{1}$: $\theta > 0$) and futility 
(attempts to reject H$_{1}$: $\theta$ = \texttt{delta} in favor of H$_{0}$: $\theta$
%TCIMACRO{\TEXTsymbol{<} }%
%BeginExpansion
$<$
%EndExpansion
\texttt{delta}). Upper and lower boundary crossing probabilities are
asymmetric. Upper boundary crossing probabilities are specified under H$_{0}$:
$\theta=0$ ($\alpha-$spending). Lower boundary crossing probabilities are
specified under H$_{1}$: $\theta$ = \texttt{delta} ($\beta-$spending).
Computation of upper boundary crossing probabilities under the null hypothesis
(Type I error) assumes the lower boundary is binding; that is, the trial 
{\em must} stop if the lower boundary is crossed. See Section~\ref{sec:statmethods}, Statistical Methods, for details on
boundary crossing probabilities for options 3 through 6.

\item[=4:] Default. Asymmetric two-sided testing with non-binding lower bound
and beta spending. Same as \texttt{test.type=3}, except that Type I error
computation assumes lower boundary is non-binding; that is, the calculation
assumes that the trial stops only if the upper boundary is crossed---it will
continue if the lower boundary is crossed. Type II error computation assumes
the trial will stop when either boundary is crossed. The effect of this is to
raise the upper boundaries after the first interim analysis relative to
\texttt{test.type=3}. See Section~\ref{sec:statmethods}, Statistical Methods, for details.

\item[=5:] Asymmetric two-sided testing with binding lower bound and lower bound
spending specified under the null hypothesis. Same as \texttt{test.type=3},
except that lower boundary crossing probabilities are specified under H$_{0}$:
$\theta=0$. See Section~\ref{sec:statmethods}, Statistical Methods, for details.

\item[=6:] Asymmetric two-sided testing with non-binding lower bound and lower
bound spending specified under the null hypothesis. Same as 
\texttt{test.type=4}, except that lower boundary crossing probabilities are 
specified under H$_{0}$: $\theta=0$. See Section~\ref{sec:statmethods}, 
Statistical Methods, for details.
\end{description}

\item \texttt{alpha} = probability value (real; 0
%TCIMACRO{\TEXTsymbol{<} }%
%BeginExpansion
$<$
%EndExpansion
\texttt{alpha}
%TCIMACRO{\TEXTsymbol{<} }%
%BeginExpansion
$<$
%EndExpansion
1; for \texttt{test.type=2}, must have \texttt{alpha}
%TCIMACRO{\TEXTsymbol{<} }%
%BeginExpansion
$<$
%EndExpansion
0.5). Specifies the total Type I error summed across all analyses. For all
design types (including symmetric, two-sided) this is the probability of
crossing the upper boundary under the null hypothesis. The default value is
$0.025$. For symmetric, two-sided testing (\texttt{test.type=2}), this 
translates into 0.05 for the combined total probability of crossing a 
lower or upper bound at any analysis. See 
Section~\ref{sec:statmethods}, Statistical Methods, for details.

\item \texttt{beta} = probability value (real; 0
%TCIMACRO{\TEXTsymbol{<} }%
%BeginExpansion
$<$
%EndExpansion
\texttt{beta}
%TCIMACRO{\TEXTsymbol{<} }%
%BeginExpansion
$<$
%EndExpansion
1 - \texttt{alpha}). Specifies the total Type II error summed across all
analyses. The default value is $0.1$, which corresponds to 90\% power.

\item \texttt{astar} = probability value (real; 0 $\leq$ \texttt{astar} 
$\leq 1 -$ \texttt{alpha}). Normally not specified. If \texttt{test.type=5} or 
\texttt{test.type=6},
\texttt{astar} specifies the probability of crossing a lower bound at all
analyses combined. This is changed to $1 - $\texttt{alpha} when the default 
value of 0 is used. Since this is the expected usage, \texttt{astar} is 
not normally specified by the user.

\item \texttt{delta} = real value ($\delta > 0$). This is the standardized
effect size used for alternative hypothesis for which the design is powered.
If \texttt{delta}
%TCIMACRO{\TEXTsymbol{>} }%
%BeginExpansion
$>$
%EndExpansion
0, sample size is determined using this effect size (see 
Section~\ref{sec:statmethods}, Statistical Methods, for details); If the 
default \texttt{delta = 0} is given, sample size is
determined by \texttt{n.fix} as noted below. Only one of \texttt{delta} and
\texttt{n.fix} need be specified by the user.

\item \texttt{n.fix} = real value (\texttt{n.fix}
%TCIMACRO{\TEXTsymbol{>} }%
%BeginExpansion
$>$
%EndExpansion
0). If \texttt{delta}%
%TCIMACRO{\TEXTsymbol{>}}%
%BeginExpansion
$>$%
%EndExpansion
0 then \texttt{n.fix} is ignored. For the default values (\texttt{delta = 0} and
\texttt{n.fix = 1}) the returned values (R$_{1}$, \ldots, R$_{K}$) for sample
sizes are actually sample size ratios compared to a fixed design with no
interim analysis, where the denominator for all of these ratios is the sample
size for a fixed design with no interim analysis based on input values of
\texttt{alpha} and \texttt{beta}. If \texttt{delta = 0} and \texttt{n.fix}
%TCIMACRO{\TEXTsymbol{>} }%
%BeginExpansion
$>$
%EndExpansion
1, think of \texttt{n.fix} as the fixed sample size of a study without interim
analysis for the given values of \texttt{alpha} and \texttt{beta}.
\texttt{gsDesign()} then inflates this value to get a maximum sample size
to give the specified Type II error/power for the specified group sequential
design. We also denote \texttt{n.fix} as $N_{fix}$ below. See 
Section~\ref{sec:statmethods}, Statistical Methods,
for details on how to calculate sample size.

\item \texttt{timing} = 1 or \texttt{c(t}$_{\mathtt{1}}$\texttt{, \ldots,
t}$_{\mathtt{k-1}}$\texttt{)} or \texttt{c(t}$_{\mathtt{1}}$\texttt{, \ldots, 
t}$_{\mathtt{k}}$\texttt{)} where 0 %
%TCIMACRO{\TEXTsymbol{<}}%
%BeginExpansion
$<$%
%EndExpansion
 \texttt{t}$_{\mathtt{1}}$ 
%TCIMACRO{\TEXTsymbol{<}}%
%BeginExpansion
$<$%
%EndExpansion
 \ldots %
%TCIMACRO{\TEXTsymbol{<}}%
%BeginExpansion
$<$%
%EndExpansion
 \texttt{t}$_{k}$ = 1. Specifies the cumulative proportionate timing for
analyses; for example, \texttt{t}$_{\mathtt{2}} = 0.4$ means the second interim 
includes the first 40\% of planned observations. For equal spacing,
\texttt{timing = 1} (default), the timing is determined by the number of
interim looks \texttt{k}.

\item \texttt{sfu} = spending function (default = \texttt{sfHSD}). The
parameter sfu specifies the upper boundary using a spending function. For
one-sided and symmetric two-sided testing (\texttt{test.type}=1, 2),
\texttt{sfu} is the only spending function required to specify spending. The
default value is \texttt{sfHSD}, which is a Hwang-Shih-DeCani spending
function. See Section~\ref{sec:spendfun}, Spending Functions, for details on
the available options.

\item \texttt{sfupar} = real value (default = $-4$) The parameter
\texttt{sfupar} specifies any parameters needed for the spending function
specified by \texttt{sfu}. The default of $-4$ when used with the default
\texttt{sfu=sfHSD} provides a conservative, O'Brien-Fleming-like upper bound.
For spending functions that do not require parameters this is ignored.
Again, see Section~\ref{sec:spendfun}, Spending Functions, for details.

\item \texttt{sfl} = spending function (default = \texttt{sfHSD}). Specifies
the spending function for lower boundary crossing probabilities. The parameter
\texttt{sfl} is ignored for one-sided testing (\texttt{test.type = 1}) or
symmetric two-sided testing (\texttt{test.type = 2}).

\item \texttt{sflpar} = real value (default = $-2$). The parameter
\texttt{sflpar} specifies any parameters needed for the spending function
specified by \texttt{sfl}. The default value of $-2$ when used with the default
\texttt{sfl=sfHSD} provides a somewhat conservative lower bound. For spending
functions that do not require parameters this is ignored.

\item \texttt{tol} = value. Specifies the stopping increment for iterative
root-finding algorithms used to derive the user-specified design. The default
value is $0.000001$; must be
%TCIMACRO{\TEXTsymbol{>}}%
%BeginExpansion
$>$%
%EndExpansion
0 and $\leq 0.1$. This should probably be ignored by the user, but is provided
as a tuning parameter.

\item \texttt{r} = positive integer up to 80 (default = 18). This is a
parameter used to set the grid size for numerical integration computations;
see Jennison and Turnbull \cite{JTBook}, Chapter 19. This should probably be
ignored by the user, but is provided as a tuning parameter.

\item \texttt{n.I} = 
%RBC equals what?
Used for re-setting bounds when timing of analyses changes
from initial design; see Section~\ref{sec:reset}, Resetting Timing of Analyses.

\item \texttt{maxn.I} = 
%RBC equals what? "maxn.I=PlanUsed" was original text...
Used for re-setting bounds when timing of analyses
changes from initial design; see Section~\ref{sec:reset}, Resetting Timing of Analyses.
\end{itemize}

\subsection{gsProbability() syntax}

\texttt{gsProbability()} computes boundary crossing probabilities and expected
sample size of a design for arbitrary user-specified treatment effects,
interim analysis times, and bounds. The value returned has class
\texttt{gsProbability} as described in 
Section~\ref{sec:objtypes}, The \texttt{gsDesign} and \texttt{gsProbability} 
Object Types. The generic call to \texttt{gsProbability()} is of the form:

\bigskip

\texttt{gsProbability(k = 0, theta, n.I, a, b, r = 18, d=NULL)}

\bigskip

The value of \texttt{theta} must always be specified. This function is
designed to be called in one of two ways. First, using a returned value from a
call to \texttt{gsDesign()} in the input parameter \texttt{d}. If \texttt{d}
is non-null, the parameterization specified there determines the output rather
than the values of \texttt{k}, \texttt{n.I}, \texttt{a}, \texttt{b}, and
\texttt{r}. If \texttt{k} is not the default of 0, the user must specify the
set of input variables \texttt{k}, \texttt{n.I}, \texttt{a}, \texttt{b}, and
\texttt{r}. In the latter form, the arguments \texttt{n.I}, \texttt{a}, and
\texttt{b}, must all be vectors with a common length \texttt{k} equaling the
number of analyses, interim and final, in the design.

The arguments are as follows:

\begin{itemize}
\item \texttt{k} = non-negative integer (default = 0). The number of
analyses, including interim and final; default value of 0. This {\em must} be 
0 if the argument \texttt{d} is non-null.

\item \texttt{theta} = vector of real values. This specifies parameter values
(standardized effect sizes) for which boundary crossing probabilities and
expected sample sizes are desired.

\item \texttt{n.I} = vector of length \texttt{k} of real values 0 %
%TCIMACRO{\TEXTsymbol{<} }%
%BeginExpansion
$<$
%EndExpansion
 \texttt{I}$_{\mathtt{1}}$%
%TCIMACRO{\TEXTsymbol{<} }%
%BeginExpansion
$<$
%EndExpansion
\texttt{I}$_{\mathtt{2}}$\ldots%
%TCIMACRO{\TEXTsymbol{<}}%
%BeginExpansion
$<$%
%EndExpansion
\texttt{I}$_{\mathtt{k}}$. Specifies the statistical information
\texttt{I}$_{\mathtt{1}}$\texttt{, \ldots, I}$_{\mathtt{k}}$ (see 
Section~\ref{sec:statmethods}, Statistical Methods)
for each analysis.

\item \texttt{a} = vector of length \texttt{k} of real values $(l_{1}%
, \ldots, l_{k})$. Specifies lower boundary values.

\item \texttt{b} = vector of length \texttt{k} of real values $(u_{1}%
, \ldots, u_{k})$. Specifies upper boundary values. For $i$ %
%TCIMACRO{\TEXTsymbol{<}}%
%BeginExpansion
$<$%
%EndExpansion
 \texttt{k} should have $l_{i} < u_{i}$. For $i = $\texttt{k}, should have
$l_{i} \leq u_{i}$.

\item \texttt{r} =  positive integer up to 80 (default = 18). Same as for
\texttt{gsDesign()}. This should probably be ignored by the user, but is
provided as a tuning parameter.

\item \texttt{d} = a returned value from a call to \texttt{gsDesign()};
values of \texttt{k}, \texttt{n.I}, \texttt{a}, \texttt{b}, and \texttt{r}
are taken from \texttt{d} when the input value of \texttt{k} is 0. Note
that if \texttt{d} is specified and \texttt{k = 0}, then the returned value 
is an object of type \texttt{gsDesign}; otherwise, the type returned is the
simpler \texttt{gsProbability} class.
\end{itemize}

\subsection{Conditional power:\ gsCP() and gsBoundCP() syntax}

The \texttt{gsCP()} function takes a given group sequential design, assumes 
an interim z-statistic at a specified interim analysis and computes 
boundary crossing probabilities at future planned analyses. The value 
returned has class \texttt{gsProbability} which contains a design based on 
observations after interim \texttt{i} that is input and probabilities of 
boundary crossing for that design for the input values of \texttt{theta}. 
These boundary crossing probabilities are the conditional probabilities of 
crossing future bounds; see Section~\ref{sec:statmethods}, Statistical Methods,
and Section~\ref{sec:detailedex}, Detailed Examples. The syntax for 
\texttt{gsCP()} takes the form:

\bigskip

\texttt{gsCP(x, theta = NULL, i = 1, zi = 0, r = 18)}

\bigskip

where \texttt{theta}, and \texttt{r} are defined as for 
\texttt{gsProbability()}. In addition, we have

\begin{itemize}
\item \texttt{x} =  a returned value from a call to \texttt{gsDesign()} or
\texttt{gsProbability()}.

\item \texttt{theta} = $\theta$ value(s) at which conditional power is to be
computed; if \texttt{NULL}, an estimated value of $\theta$ based on the interim test
statistic (\texttt{zi/sqrt(x\$n.I[i]}) as well as at \texttt{x\$theta} is computed.

\item \texttt{i} = a specified interim analysis; must be a positive integer
less than the total number of analyses specified by \texttt{x\$k}.

\item \texttt{zi} = the interim test statistic value at analysis \texttt{i};
must be in the interval 
\texttt{x\$lower\$bound[i]} $\leq$ \texttt{zi} $\leq$ \texttt{x\$upper\$bound[i]}.

\item \texttt{r} = positive integer up to 80 (default = 18). Same as for
\texttt{gsDesign()}. This should probably be ignored by the user, but is
provided as a tuning parameter.
\end{itemize}

The \texttt{gsBoundCP()} function computes the total probability of crossing 
future upper bounds given an interim test statistic at an interim bound. For 
each interim boundary 
%RBC: looks like something is missing here, perhaps gsBoundCP() assumes?
assumes an interim test statistic at the boundary and computes the
probability of crossing any of the later upper boundaries. The returned value
is a list of two vectors, \texttt{cplo} and \texttt{cphi}, which contain
conditional power at each interim analysis based on an interim test statistic
at the low and high boundaries, respectively. The syntax for
\texttt{gsBoundCP()} takes the form:

\bigskip

\texttt{gsBoundCP(x, theta = "thetahat", r = 18)}

\begin{itemize}
\item \texttt{x} = a returned value from a call to \texttt{gsDesign()} or
\texttt{gsProbability()}.

\item \texttt{theta} = 
%RBC would be nice to say what this is...
if \texttt{"thetahat"} and
\texttt{class(x)=="gsDesign"}, conditional power computations for each
boundary value are computed using estimated treatment effect assuming a test
statistic at that boundary ($\mathtt{zi}/sqrt(\mathtt{x\$n.I[i]}$ at analysis 
\texttt{i}, interim test
statistic \texttt{zi} and interim sample size/statistical information of 
\texttt{x\$n.I[i]}).
Otherwise, conditional power is computed assuming the input scalar value 
\texttt{theta}.

\item \texttt{r} = positive integer up to 80 (default = 18). Same as for
\texttt{gsDesign()}. This should probably be ignored by the user, but is
provided as a tuning parameter.
\end{itemize}

\subsection{Binomial distribution: FarrMannSS(), MandNtest(), MandNsim()
syntax}

All of these routines are designed for two-arm binomial trials and use
asymptotic approximations. They may be used for superiority or non-inferiority
trials. \texttt{FarrMannSS()} computes sample size using the method of
Farrington and Manning (1990). \texttt{MandNTest()} computes a Z- or
Chi-square-statistic that compares two binomial event rates using the method
of Miettinen and Nurminen (1980). \texttt{MandNsim()} performs simulations to
estimate the power for a Miettinin and Nurminen (1980) test.

\bigskip

\texttt{FarrMannSS(p1, p2, fraction = 0.5, alpha = 0.05, power = 0.8, beta=0,
delta0=0, }

\qquad\texttt{ratio=0, sided=2, outtype=2) }

\texttt{MandNTest(x1,x2,n1,n2,delta0=0,testtype="Chisq",adj=1)}

\texttt{MandNsim(p1, p2, n1, n2, alpha = 0.05, delta0=0, nsim = 10000,
testtype="Chisq", adj=1)}

\bigskip

\begin{itemize}
\item \texttt{p1} = event rate in group 1

\item \texttt{p2} = event rate in group 2

\item \texttt{fraction} = fraction of observations in group 1

\item \texttt{alpha} = type I error; see sided below to distinguish between
1- and 2-sided tests

\item \texttt{power} = the desired probability of detecting a difference

\item \texttt{beta} = type II error; used instead of \texttt{power} when non-zero

\item \texttt{delta0} = The absolute difference in event rates that is
considered unacceptable.

\item \texttt{ratio} = sample size ratio for group 2 divided by group 1; used
instead of fraction when non-zero

\item \texttt{sided} = 2 for 2-sided test, 1 for 1-sided test

\item \texttt{outtype} = 2 (default) returns sample size for each group
(\texttt{n1}, \texttt{n2}); otherwise returns \texttt{n1+n2}

\item \texttt{x1} = Number of "successes" in the control group

\item \texttt{x2} = Number of "successes" in the experimental group

\item \texttt{n1} = Number of observations in the control group

\item \texttt{n2} = Number of observations in the experimental group

\item \texttt{testtype} =  
%RBC--again, would be nice to say what this is--what are the allowable inputs..
If \texttt{"Chisq"} (default), a Miettinen and
Nurminen chi-square statistic for a $2 \times 2$ table (no continuity 
correction) is used. Otherwise, the difference in event rates divided by its 
standard error under the null hypothesis is used.

\item \texttt{adj} = 
%RBC--what types of input are allowable ?
With \texttt{adj=1}, the standard variance for a
Miettinen and Nurminen test statistic is used. This includes a factor of
n/(n-1) where n is the total sample size. If \texttt{adj} is not 1, this
factor is not applied. See details.%RBC--where are these details?

\item \texttt{nsim} = The number of simulations to be performed in
\texttt{MandNSim()}
\end{itemize}

\texttt{MandNTest()} and \texttt{MandNsim()} each return a vector of either
Chi-square or Z test statistics. These may be compared to an appropriate
cutoff point (for example, \texttt{qnorm(.975)} or \texttt{qchisq(.95,1)}).

With the default \texttt{outtype=2}, \texttt{FarrMannSS()} returns a list
containing two vectors \texttt{n1} and \texttt{n2} containing sample sizes for
groups 1 and 2, respectively. With \texttt{outtype=1}, a vector of total
sample sizes is returned. With \texttt{outtype=3}, \texttt{FarrMannSS()}
returns a list as follows:

\bigskip

\begin{itemize}
\item \texttt{n}: A vector with total samples size required for each event
rate comparison specified

\item \texttt{n1}: A vector of sample sizes for group 1 for each event rate
comparison specified

\item \texttt{n2}: A vector of sample sizes for group 2 for each event rate
comparison specified

\item \texttt{sigma0}: A vector containing the variance of the treatment
effect difference under the null hypothesis

\item \texttt{sigma1}: A vector containing the variance of the treatment
effect difference under the alternative hypothesis

\item \texttt{p1}: As input

\item \texttt{p2}: As input

\item \texttt{pbar}: Returned only for superiority testing 
(\texttt{delta0 = 0}, the weighted average of \texttt{p1} and 
\texttt{p2} using weights \texttt{n1} and \texttt{n2}

\item \texttt{p10}: group 1 treatment effect used for null hypothesis

\item \texttt{p20}: group 2 treatment effect used for null hypothesis
\end{itemize}

