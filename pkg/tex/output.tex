\section{Formatted Output\label{sec:formatted}}

\texttt{gsDesign()} returns an object of the class \texttt{gsDesign}.
\texttt{gsProbability()} returns an object of class gsDesign if a \texttt{gsDesign} class object is passed in or of class \texttt{gsProbability}, if not. 
The standard R functions \texttt{print()}, and \texttt{plot()} are extended 
to work for both the \texttt{gsDesign} and the \texttt{gsProbability} classes. 
Note also that \texttt{summary()} prints a brief summary of either object 
type if you need a reminder of what is
in a class. The \texttt{print()} function for each object class has a single
argument and is implemented through the functions \texttt{print.gsDesign()}
and \texttt{print.gsProbability()}. The \texttt{plot()} functions for 
\texttt{gsDesign} and \texttt{gsProbability} objects have a second argument 
that specifies which type of
plot is to be made; these functions are implemented with the functions
\texttt{plot.gsDesign()} and \texttt{plot.gsProbability()}, respectively.
There are seven plot types, six of which are available for both classes, 
that are specified through the input variable \texttt{plottype}:

\bigskip

\begin{itemize}
\item \texttt{1} or \texttt{"Z"}: boundary plots (default if
\texttt{class(x)="gsDesign"})

\item \texttt{2} or \texttt{"power"}: boundary crossing probability plots
(default if \texttt{class(x)="gsProbability"})

\item \texttt{3} or \texttt{"thetahat"}: estimated treatment effect at boundaries

\item \texttt{4 }or\texttt{\ "CP"}: conditional power at boundaries

\item \texttt{5} or \texttt{"sf"}: spending function plot (available only if
\texttt{class(x)=="gsDesign"})

\item \texttt{6} or \texttt{"ASN"} or \texttt{"N"}: expected sample size
plot

\item \texttt{7"B"} or \texttt{"B-val"} or \texttt{"B-value"}:  B-values at boundaries
\end{itemize}

\bigskip


Further details of the \texttt{print()}, and \texttt{plot()} functions are
provided in Section~\ref{sec:detailedex}, Detailed Examples, and in the help 
file for plotting in the gsDesign
package. Since the plot functions have multiple arguments, the use of these
arguments is formally specified here.

\bigskip

\texttt{plot.gsDesign(x,plottype=1, main="Default", ylab="Default",
xlab="Default")}

\texttt{plot.gsProbability(x,plottype=2, xval="Default", main="Default",}

\texttt{\qquad ylab="Default", xlab="Default")}

\bigskip

\begin{itemize}
\item \texttt{x}: for \texttt{plot.gsDesign()}, this is an object in the
\texttt{gsDesign} class; for \texttt{plot.gsProbability()}, this is an 
object in the \texttt{gsProbability} class.

\item \texttt{plottype}: described above.

\item \texttt{theta}: used for \texttt{plottype} equal to 2, 4, or 6; normally 
defaults are adequate. See the help files for details.

\item \texttt{ses=TRUE}: applies only when \texttt{plottype = 3} and 
\texttt{class(x) == "gsDesign"};
indicates that estimated standardized effect size at the boundary
($\hat{\theta}/\delta$) is to be plotted rather than the actual estimate
($\hat{\theta}$.

\item \texttt{xval="Default"}: effective only when \texttt{plottype} equals 2 
or 6.  Appropriately scaled (reparameterized) values for x-axis for power and
expected sample size graphs; see the help file for details.
\end{itemize}

