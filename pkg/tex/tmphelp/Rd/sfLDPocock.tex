\HeaderA{sfLDOF}{4.4: Lan-DeMets Spending function overview}{sfLDOF}
\aliasA{sfLDPocock}{sfLDOF}{sfLDPocock}
\keyword{design}{sfLDOF}
\begin{Description}\relax
Lan and DeMets (1983) first published the method of using spending functions to set boundaries for group sequential trials.
In this publication they proposed two specific spending functions:
one to approximate an O'Brien-Fleming design and the other to approximate a Pocock design.
Both of these spending functions are available here, mainly for historical purposes.
Neither requires a parameter.
\end{Description}
\begin{Usage}
\begin{verbatim}
sfLDOF(alpha, t, param)
sfLDPocock(alpha, t, param)
\end{verbatim}
\end{Usage}
\begin{Arguments}
\begin{ldescription}
\item[\code{alpha}] Real value \eqn{> 0}{} and no more than 1. Normally, 
\code{alpha=0.025} for one-sided Type I error specification
or \code{alpha=0.1} for Type II error specification. However, this could be set to 1 if for descriptive purposes
you wish to see the proportion of spending as a function of the proportion of sample size/information.
\item[\code{t}] A vector of points with increasing values from 0 to 1, inclusive. Values of the proportion of 
sample size/information for which the spending function will be computed.
\item[\code{param}] This parameter is not used and need not be specified. It is here so that the calling sequence conforms 
the to the standard for spending functions used with \code{gsDesign()}.
\end{ldescription}
\end{Arguments}
\begin{Details}\relax
The Lan-DeMets (1983) spending function to approximate an
O'Brien-Fleming bound is implemented in the function (\code{sfLDOF()}):
\deqn{f(t; \alpha)=2-2\Phi\left(\Phi^{-1}(1-\alpha/2)/ t^{1/2}\right).}{f(t; alpha)=2-2*Phi(Phi^(-1)(1-alpha/2)/t^(1/2)\right).}
The Lan-DeMets (1983) spending function to approximate a Pocock design is implemented in the function \code{sfLDPocock()}:
\deqn{f(t;\alpha)=ln(1+(e-1)t).}{f(t;alpha)=ln(1+(e-1)t).}
As shown in examples below, other spending functions can be used to get as good or better approximations to Pocock and
O'Brien-Fleming bounds. In particular, O'Brien-Fleming bounds can be closely approximated using \code{\LinkA{sfExponential}{sfExponential}}.
\end{Details}
\begin{Value}
An object of type \code{spendfn}. See spending functions for further details.
\end{Value}
\begin{Note}\relax
The manual is not linked to this help file, but is available in library/gsdesign/doc/gsDesignManual.pdf
in the directory where R is installed.
\end{Note}
\begin{Author}\relax
Keaven Anderson \email{keaven\_anderson@merck.}
\end{Author}
\begin{References}\relax
Jennison C and Turnbull BW (2000), \emph{Group Sequential Methods with Applications to Clinical Trials}.
Boca Raton: Chapman and Hall.

Lan, KKG and DeMets, DL (1983), Discrete sequential boundaries for clinical trials. \emph{Biometrika};70: 659-663.
\end{References}
\begin{SeeAlso}\relax
\LinkA{Spending function overview}{Spending function overview}, \code{\LinkA{gsDesign}{gsDesign}}, \LinkA{gsDesign package overview}{gsDesign package overview}
\end{SeeAlso}
\begin{Examples}
\begin{ExampleCode}
# 2-sided,  symmetric 6-analysis trial Pocock
# spending function approximation 
gsDesign(k=6, sfu=sfLDPocock, test.type=2)$upper$bound

# show actual Pocock design
gsDesign(k=6, sfu="Pocock", test.type=2)$upper$bound

# approximate Pocock again using a standard
# Hwang-Shih-DeCani approximation
gsDesign(k=6, sfu=sfHSD, sfupar=1, test.type=2)$upper$bound

# use 'best' Hwang-Shih-DeCani approximation for Pocock,  k=6;
# see manual for details
gsDesign(k=6, sfu=sfHSD, sfupar=1.3354376, test.type=2)$upper$bound

# 2-sided, symmetric 6-analysis trial
# O'Brien-Fleming spending function approximation 
gsDesign(k=6, sfu=sfLDOF, test.type=2)$upper$bound

# show actual O'Brien-Fleming bound
gsDesign(k=6, sfu="OF", test.type=2)$upper$bound

# approximate again using a standard Hwang-Shih-DeCani 
# approximation to O'Brien-Fleming
x<-gsDesign(k=6, test.type=2)
x$upper$bound
x$upper$param

# use 'best' exponential approximation for k=6; see manual for details
gsDesign(k=6, sfu=sfExponential, sfupar=0.7849295,
         test.type=2)$upper$bound
\end{ExampleCode}
\end{Examples}

