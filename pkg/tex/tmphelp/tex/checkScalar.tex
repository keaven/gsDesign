\HeaderA{checkScalar}{verify variable properties}{checkScalar}
\aliasA{checkLengths}{checkScalar}{checkLengths}
\aliasA{checkRange}{checkScalar}{checkRange}
\aliasA{checkVector}{checkScalar}{checkVector}
\aliasA{isInteger}{checkScalar}{isInteger}
\keyword{programming}{checkScalar}
\begin{Description}\relax
Utility functions to verify an objects's properties including whether it is a scalar or vector,
the class, the length, and (if numeric) whether the range of values is on a specified interval. Additionally,
the \code{checkLengths} function can be used to ensure that all the supplied inputs have equal lengths.
\end{Description}
\begin{Usage}
\begin{verbatim}
isInteger(x)
checkScalar(x, isType = "numeric", ...)
checkVector(x, isType = "numeric", ..., length=NULL) 
checkRange(x, interval = 0:1, inclusion = c(TRUE, TRUE), varname = deparse(substitute(x)))
checkLengths(...)
\end{verbatim}
\end{Usage}
\begin{Arguments}
\begin{ldescription}
\item[\code{x}] any object.
\item[\code{isType}] character string defining the class that the input object is expected to be.
\item[\code{length}] integer specifying the expected length of the object in the case it is a vector. If \code{length=NULL}, the default,
then no length check is performed.
\item[\code{interval}] two-element numeric vector defining the interval over which the input object is expected to be contained. 
Use the \code{inclusion} argument to define the boundary behavior.
\item[\code{inclusion}] two-element logical vector defining the boundary behavior of the specified interval. A \code{TRUE} value
denotes inclusion of the corresponding boundary. For example, if \code{interval=c(3,6)} and \code{inclusion=c(FALSE,TRUE)},
then all the values of the input object are verified to be on the interval (3,6].
\item[\code{varname}] character string defining the name of the input variable as sent into the function by the caller. 
This is used primarily as a mechanism to specify the name of the variable being tested when \code{checkRange} is being called
within a function.
\item[\code{...}] For the \code{\LinkA{checkScalar}{checkScalar}} and \code{\LinkA{checkVector}{checkVector}} functions, this input represents additional 
arguments sent directly to the \code{\LinkA{checkRange}{checkRange}} function. For the \code{\LinkA{checkLengths}{checkLengths}} function, this input
represents the arguments to check for equal lengths.
\end{ldescription}
\end{Arguments}
\begin{Details}\relax
\code{isInteger} is similar to \code{\LinkA{is.integer}{is.integer}} except that \code{isInteger(1)} returns \code{TRUE} whereas \code{is.integer(1)} returns \code{FALSE}.

\code{checkScalar} is used to verify that the input object is a scalar as well as the other properties specified above. 

\code{checkVector} is used to verify that the input object is an atomic vector as well as the other properties as defined above.

\code{checkRange} is used to check whether the numeric input object's values reside on the specified interval. 
If any of the values are outside the specified interval, a \code{FALSE} is returned.

\code{checkLength} is used to check whether all of the supplied inputs have equal lengths.
\end{Details}
\begin{Examples}
\begin{ExampleCode}
# check whether input is an integer
isInteger(1)
isInteger(1:5)
try(isInteger("abc")) # expect error

# check whether input is an integer scalar
checkScalar(3, "integer")

# check whether input is an integer scalar that resides 
# on the interval on [3, 6]. Then test for interval (3, 6].
checkScalar(3, "integer", c(3,6))
try(checkScalar(3, "integer", c(3,6), c(FALSE, TRUE))) # expect error

# check whether the input is an atomic vector of class numeric,
# of length 3, and whose value all reside on the interval [1, 10)
x <- c(3, pi, exp(1))
checkVector(x, "numeric", c(1, 10), c(TRUE, FALSE), length=3)

# do the same but change the expected length
try(checkVector(x, "numeric", c(1, 10), c(TRUE, FALSE), length=2)) # expect error

# create faux function to check input variable
foo <- function(moo) checkVector(moo, "character")
foo(letters)
try(foo(1:5)) # expect error with function and argument name in message

# check for equal lengths of various inputs
checkLengths(1:2, 2:3, 3:4)
try(checkLengths(1,2,3,4:5)) # expect error
\end{ExampleCode}
\end{Examples}

