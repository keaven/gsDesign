\HeaderA{gsProbability}{2.2: Boundary Crossing Probabilities}{gsProbability}
\aliasA{print.gsProbability}{gsProbability}{print.gsProbability}
\keyword{design}{gsProbability}
\begin{Description}\relax
Computes power/Type I error and expected sample size for a group sequential design
across a selected set of parameter values for a given set of analyses and boundaries.
The print function has been extended using \code{print.gsProbability} to print \code{gsProbability} objects; see examples.
\end{Description}
\begin{Usage}
\begin{verbatim}
gsProbability(k=0, theta, n.I, a, b, r=18, d=NULL)
\end{verbatim}
\end{Usage}
\begin{Arguments}
\begin{ldescription}
\item[\code{k}] Number of analyses planned, including interim and final.
\item[\code{theta}] Vector of standardized effect sizes for which boundary crossing probabilities are to be computed.
\item[\code{n.I}] Sample size or relative sample size at analyses; vector of length k. See \code{\LinkA{gsDesign}{gsDesign}} and manual.
\item[\code{a}] Lower bound cutoffs (z-values) for futility or harm at each analysis, vector of length k.
\item[\code{b}] Upper bound cutoffs (z-values) for futility at each analysis; vector of length k.
\item[\code{r}] Control for grid as in Jennison and Turnbull (2000); default is 18, range is 1 to 80.
Normally this will not be changed by the user.
\item[\code{d}] If not \code{NULL}, this should be an object of type \code{gsDesign} returned by a call to \code{gsDesign()}. 
When this is specified, the values of \code{k}, \code{n.I}, \code{a}, \code{b}, and \code{r} will be obtained from \code{d} and
only \code{theta} needs to be specified by the user.
\end{ldescription}
\end{Arguments}
\begin{Details}\relax
Depending on the calling sequence, an object of class \code{gsProbability} or class \code{gsDesign} is returned.
If it is of class \code{gsDesign} then the members of the object will be the same as described in \code{\LinkA{gsDesign}{gsDesign}}.
If \code{d} is input as \code{NULL} (the default), all other arguments (other than \code{r}) must be specified 
and an object of class \code{gsProbability} is returned.
If \code{d} is passed as an object of class \code{gsProbability} or \code{gsDesign} the only other argument required is \code{theta};
the object returned has the same class as the input \code{d}.
On output, the values of \code{theta} input to \code{gsProbability} will be the parameter values for which the
design is characterized.
\end{Details}
\begin{Value}
\begin{ldescription}
\item[\code{k}] As input.
\item[\code{theta}] As input.
\item[\code{n.I}] As input.
\item[\code{lower}] A list containing two elements: \code{bound} is as input in \code{a} and \code{prob} is a matrix of boundary 
crossing probabilities. Element \code{i,j} contains the boundary crossing probability at analysis \code{i} for the \code{j}-th
element of \code{theta} input. All boundary crossing is assumed to be binding for this computation; 
that is, the trial must stop if a boundary is crossed.
\item[\code{upper}] A list of the same form as \code{lower} containing the upper bound and upper boundary crossing probabilities.
\item[\code{en}] A vector of the same length as \code{theta} containing expected sample sizes for the trial design
corresponding to each value in the vector \code{theta}.
\item[\code{r}] As input.
\end{ldescription}
\end{Value}
\begin{Note}\relax
The manual is not linked to this help file, but is available in library/gsdesign/doc/gsDesignManual.pdf
in the directory where R is installed.
\end{Note}
\begin{Author}\relax
Keaven Anderson \email{keaven\_anderson@merck.}
\end{Author}
\begin{References}\relax
Jennison C and Turnbull BW (2000), \emph{Group Sequential Methods with Applications to Clinical Trials}.
Boca Raton: Chapman and Hall.
\end{References}
\begin{SeeAlso}\relax
\LinkA{Plots for group sequential designs}{Plots for group sequential designs}, \code{\LinkA{gsDesign}{gsDesign}}, \LinkA{gsDesign package overview}{gsDesign package overview}
\end{SeeAlso}
\begin{Examples}
\begin{ExampleCode}
# making a gsDesign object first may be easiest...
x <- gsDesign()

# take a look at it
x

# default plot for gsDesign object shows boundaries
plot(x)

# plottype=2 shows boundary crossing probabilities
plot(x, plottype=2)

# now add boundary crossing probabilities and 
# expected sample size for more theta values
y <- gsProbability(d=x, theta=x$delta*seq(0, 2, .25))
class(y)

# note that "y" below is equivalent to print(y) and
# print.gsProbability(y)
y

# the plot does not change from before since this is a
# gsDesign object; note that theta/delta is on x axis
plot(y, plottype=2)

# now let's see what happens with a gsProbability object
z <- gsProbability(k=3, a=x$lower$bound, b=x$upper$bound, 
    n.I=x$n.I, theta=x$delta*seq(0, 2, .25))

# with the above form,  the results is a gsProbability object
class(z)
z

# default plottype is now 2
# this is the same range for theta, but plot now has theta on x axis
plot(z)
\end{ExampleCode}
\end{Examples}

